\documentclass{article}
\usepackage{graphicx}
\usepackage{xcolor}
\usepackage{amsmath}
\usepackage{multicol}
\begin{document}
\begin{titlepage}
\begin{center}
	\vspace*{1cm}

	\Huge
	\textbf{Internet of Things Application}

	\vspace{0.5cm}
	\LARGE
	Analysis of Project and Team Plan

	\vspace{1.5cm}
    \Large

	\textbf{}\\
	\textbf{}\\
	\textbf{}\\
	\textbf{}\\
	\textbf{Quin'darius Lyles-Woods}\\
    

	\vfill
	\LARGE
	\vspace{0.8cm}

	\includegraphics[width=\textwidth]{kennesawlogo}

	\vspace{0.8cm}

	\Large
	\textbf{Professor Ken Hoganson} 			\\
    \textbf{CS 4850 Senior Project/Capstone} \\
	Department of Computer Science   \\
	Kennesaw State University       \\
	1100 South Marietta Pkwy SE     \\
	Marietta, GA 30060              \\

	\vspace{1cm}

\end{center}
\end{titlepage}


\section{Description}

An Internet of Things device that will record the inputs and outputs of the items within it. 
Correctly identifying the items over time and telling how much is used after each use case.
The sensors will have an accompanying application to help identify the weights of each item.
This will allow the user to sign in the items once and never have to worry about them again.
After the Internet of Things device will be able to understand when an object is empty.
That is the gist of the use case for the application. 

The benefits of this application will be as follows. 
Low cost application for remotely monitoring the quantity of food stuffs and other household goods.
This can then be then integrated with numerous other APIs that will cause actions automatically.
That will allow for automatic order of food stuffs or other items and even just general notification and information gathering. 

\section{Components}

An microprocessor with accompanying sensors that detect weight will be one of the key components. 
Another component to this Internet of Things Application will the Mobile Application that will be receiving the data from the previously mentioned device.
Having the sensors detecting the weight will be augmented with presence sensors that will be riddled throughout the shelf's service.
With a combination of these three data points we have a large chance of successfully understanding the inputs and outputs of the shelf.

\subsection{Skills Required}

The skills required for this application are tightly coupled with the Components requirements.
The skill for the microprocessor development will require embedded systems knowledge.
Particularly for being able to transmit the information from the microprocessor to any other system will be key to the success of the project.
Along with the knowledge need for the microprocessor the information for choosing the correct sensor will also be needed.
With choosing the correct sensory devices we will need to understand what will be the best sensory location and pressure gauges for this application.

That sums up the skills needed with our currently knowledge of the embedded side of this application. 
With the Internet of Things you have more than just an embedded side of the application. 
You will also need the ability to compile all the information the device is outputting. 
The embedding device could be used to understand that data but that will couple failure points closer than we would like.
For this reason we will be having all the data be processed on the mobile application.
This will need the skills of various mobile programming languages.
Such as Swift and Kotlin for the Android and iOS applications.

\subsection{Embedded System}
An microprocessor with accompanying sensors that detect weight will be one of the key components. 
\subsubsection{Skills Required}
The skill for the microprocessor development will require embedded systems knowledge.
Particularly for being able to transmit the information from the microprocessor to any other system will be key to the success of the project.
Along with the knowledge need for the microprocessor the information for choosing the correct sensor will also be needed.
With choosing the correct sensory devices we will need to understand what will be the best sensory location and pressure gauges for this application.
\subsubsection{Team Members}
Leading the development will be Genasie Gaige.
Help with development will be provided by the rest of team.


\subsection{Database}

With the database we will be able to use a very lightweight database that is given within most mobile operating systems. 
The database that we will be using will be the Sqlite Database. 
Within this we will be able to iterate very quickly with testing.
There will be no time wasted on testing this application because we will be able to test mock databases from our own computers in a light weight fashion.
Sqlite is a simple database and the entire team will be able to caught up to speed with it to the degree of their use in the application development process fairly quickly.

\subsubsection{Skills Required}
The skills needed for the database development will be listed as follows.
You will need to be able to write SQL queries to test the database.
After being able to probably test the imports of the database you will need to be able to get the statement made during testing ported to the appropriate architectures.

\subsubsection{Team Members}
Leading the development will be Hai Tu.
Help with development will be provided by the rest of team.




\subsection{Application Programming Interface}

This application will require the device to have input and output through an Application Programming Interface.
This will only the mobile application to receive information from the embedded devices and accompanying sensors.
With this it will facilitate on device processing of the raw data received from the embedded device.

\subsubsection{Skills Required}

The skills required for the application programming interface are as follows. 
First and foremost the embedded environment will need a way to spin up a possible web server. 
The server will be set up to send HTTP Request, through a restful interface.
The skills needed mainly will be networking and embedded systems knowledge.

\subsubsection{Team Members}
The development will be provided by the entire team.

\subsection{Mobile Application}
The mobile application will be a vital part to the success of the application.
Using the mobile application the data from the embedded devices will be compiled.
After the data is compiled it will be shown to the user in such a way the data is easily understood.
\subsubsection{Skills Required}
The skills need for building the mobile application will be listed as follows.
For the iOS application the developer will need to be able to build with the Swift language.
For the Android Application the develop will need to be able to build with the Kotlin language.
\subsubsection{Team Members}
Leading the development will be Quindarius Lyles-Woods.
Help with development will be provided by the rest of team.

\section{GUI Mock-ups}
\includegraphics[height = \textwidth/2]{images/Add Items.png}
\includegraphics[height = \textwidth/2]{images/Frame\ 1.png}
\includegraphics[height = \textwidth/2]{images/Grocery\ List.png}
\includegraphics[height = \textwidth/2]{images/Home.png}
\includegraphics[height = \textwidth/2]{images/Items\ List.png}
\includegraphics[height = \textwidth/2]{images/Manual.png}
\includegraphics[height = \textwidth/2]{images/Scan.png}


\end{document}
